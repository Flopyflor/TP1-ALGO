% !TEX encoding = UTF-8 Unicode
\documentclass[10pt,a4paper]{article}

\usepackage[spanish,activeacute,es-tabla]{babel}
\usepackage[utf8]{inputenc}
\usepackage{ifthen}
\usepackage{listings}
\usepackage{dsfont}
\usepackage{subcaption}
\usepackage{amsmath}
\usepackage[strict]{changepage}
\usepackage[top=1cm,bottom=2cm,left=1cm,right=1cm]{geometry}%
\usepackage{color}%
\newcommand{\tocarEspacios}{%
	\addtolength{\leftskip}{3em}%
	\setlength{\parindent}{0em}%
}

% Especificacion de procs

\newcommand{\In}{\textsf{in }}
\newcommand{\Out}{\textsf{out }}
\newcommand{\Inout}{\textsf{inout }}

\newcommand{\encabezadoDeProc}[4]{%
	% Ponemos la palabrita problema en tt
	%  \noindent%
	{\normalfont\bfseries\ttfamily proc}%
	% Ponemos el nombre del problema
	\ %
	{\normalfont\ttfamily #2}%
	\
	% Ponemos los parametros
	(#3)%
	\ifthenelse{\equal{#4}{}}{}{%
		% Por ultimo, va el tipo del resultado
		\ : #4}
}

\newenvironment{proc}[4][res]{%
	
	% El parametro 1 (opcional) es el nombre del resultado
	% El parametro 2 es el nombre del problema
	% El parametro 3 son los parametros
	% El parametro 4 es el tipo del resultado
	% Preambulo del ambiente problema
	% Tenemos que definir los comandos requiere, asegura, modifica y aux
	\newcommand{\requiere}[2][]{%
		{\normalfont\bfseries\ttfamily requiere}%
		\ifthenelse{\equal{##1}{}}{}{\ {\normalfont\ttfamily ##1} :}\ %
		\{\ensuremath{##2}\}%
		{\normalfont\bfseries\,\par}%
	}
	\newcommand{\asegura}[2][]{%
		{\normalfont\bfseries\ttfamily asegura}%
		\ifthenelse{\equal{##1}{}}{}{\ {\normalfont\ttfamily ##1} :}\
		\{\ensuremath{##2}\}%
		{\normalfont\bfseries\,\par}%
	}
	\renewcommand{\aux}[4]{%
		{\normalfont\bfseries\ttfamily aux\ }%
		{\normalfont\ttfamily ##1}%
		\ifthenelse{\equal{##2}{}}{}{\ (##2)}\ : ##3\, = \ensuremath{##4}%
		{\normalfont\bfseries\,;\par}%
	}
	\renewcommand{\pred}[3]{%
		{\normalfont\bfseries\ttfamily pred }%
		{\normalfont\ttfamily ##1}%
		\ifthenelse{\equal{##2}{}}{}{\ (##2) }%
		\{%
		\begin{adjustwidth}{+5em}{}
			\ensuremath{##3}
		\end{adjustwidth}
		\}%
		{\normalfont\bfseries\,\par}%
	}
	
	\newcommand{\res}{#1}
	\vspace{1ex}
	\noindent
	\encabezadoDeProc{#1}{#2}{#3}{#4}
	% Abrimos la llave
	\par%
	\tocarEspacios
}
{
	% Cerramos la llave
	\vspace{1ex}
}

\newcommand{\aux}[4]{%
	{\normalfont\bfseries\ttfamily\noindent aux\ }%
	{\normalfont\ttfamily #1}%
	\ifthenelse{\equal{#2}{}}{}{\ (#2)}\ : #3\, = \ensuremath{#4}%
	{\normalfont\bfseries\,;\par}%
}

\newcommand{\pred}[3]{%
	{\normalfont\bfseries\ttfamily\noindent pred }%
	{\normalfont\ttfamily #1}%
	\ifthenelse{\equal{#2}{}}{}{\ (#2) }%
	\{%
	\begin{adjustwidth}{+2em}{}
		\ensuremath{#3}
	\end{adjustwidth}
	\}%
	{\normalfont\bfseries\,\par}%
}

% Tipos

\newcommand{\nat}{\ensuremath{\mathds{N}}}
\newcommand{\ent}{\ensuremath{\mathds{Z}}}
\newcommand{\float}{\ensuremath{\mathds{R}}}
\newcommand{\bool}{\ensuremath{\mathsf{Bool}}}
\newcommand{\cha}{\ensuremath{\mathsf{Char}}}
\newcommand{\str}{\ensuremath{\mathsf{String}}}

% Logica

\newcommand{\True}{\ensuremath{\mathrm{true}}}
\newcommand{\False}{\ensuremath{\mathrm{false}}}
\newcommand{\Then}{\ensuremath{\rightarrow}}
\newcommand{\Iff}{\ensuremath{\leftrightarrow}}
\newcommand{\implica}{\ensuremath{\longrightarrow}}
\newcommand{\IfThenElse}[3]{\ensuremath{\mathsf{if}\ #1\ \mathsf{then}\ #2\ \mathsf{else}\ #3\ \mathsf{fi}}}
\newcommand{\yLuego}{\land _L}
\newcommand{\oLuego}{\lor _L}
\newcommand{\implicaLuego}{\implica _L}

\newcommand{\cuantificador}[5]{%
	\ensuremath{(#2 #3: #4)\ (%
		\ifthenelse{\equal{#1}{unalinea}}{
			#5
		}{
			$ % exiting math mode
			\begin{adjustwidth}{+2em}{}
				$#5$%
			\end{adjustwidth}%
			$ % entering math mode
		}
		)}
}

\newcommand{\existe}[4][]{%
	\cuantificador{#1}{\exists}{#2}{#3}{#4}
}
\newcommand{\paraTodo}[4][]{%
	\cuantificador{#1}{\forall}{#2}{#3}{#4}
}

%listas

\newcommand{\TLista}[1]{\ensuremath{seq \langle #1\rangle}}
\newcommand{\lvacia}{\ensuremath{[\ ]}}
\newcommand{\lv}{\ensuremath{[\ ]}}
\newcommand{\longitud}[1]{\ensuremath{|#1|}}
\newcommand{\cons}[1]{\ensuremath{\mathsf{addFirst}}(#1)}
\newcommand{\indice}[1]{\ensuremath{\mathsf{indice}}(#1)}
\newcommand{\conc}[1]{\ensuremath{\mathsf{concat}}(#1)}
\newcommand{\cab}[1]{\ensuremath{\mathsf{head}}(#1)}
\newcommand{\cola}[1]{\ensuremath{\mathsf{tail}}(#1)}
\newcommand{\sub}[1]{\ensuremath{\mathsf{subseq}}(#1)}
\newcommand{\en}[1]{\ensuremath{\mathsf{en}}(#1)}
\newcommand{\cuenta}[2]{\mathsf{cuenta}\ensuremath{(#1, #2)}}
\newcommand{\suma}[1]{\mathsf{suma}(#1)}
\newcommand{\twodots}{\ensuremath{\mathrm{..}}}
\newcommand{\masmas}{\ensuremath{++}}
\newcommand{\matriz}[1]{\TLista{\TLista{#1}}}
\newcommand{\seqchar}{\TLista{\cha}}

\renewcommand{\lstlistingname}{Código}
\lstset{% general command to set parameter(s)
	language=Java,
	morekeywords={endif, endwhile, skip},
	basewidth={0.47em,0.40em},
	columns=fixed, fontadjust, resetmargins, xrightmargin=5pt, xleftmargin=15pt,
	flexiblecolumns=false, tabsize=4, breaklines, breakatwhitespace=false, extendedchars=true,
	numbers=left, numberstyle=\tiny, stepnumber=1, numbersep=9pt,
	frame=l, framesep=3pt,
	captionpos=b,
}

\usepackage{caratula} % Version modificada para usar las macros de algo1 de ~> https://github.com/bcardiff/dc-tex

\titulo{Trabajo práctico 1: Especificación y WP}
\subtitulo{“Fondo Monetario Común”}

\fecha{\today}

\materia{Algoritmos y Estructuras de Datos/ Ex Algo 2}
\grupo{losdecompusongentedebien}

\integrante{Allami, Florencia}{484/23}{allamiflorencia@gmail.com}
\integrante{Bartoli, Matilda}{175/23}{bartoli.matilda@gmail.com}
\integrante{Simon, Felipe Pedro}{135/23}{felipe.p.simon@gmail.com}
\integrante{Sosa, Luciano}{1011/23}{sosaluciano020104@gmail.com}
% Pongan cuantos integrantes quieran

\graphicspath{{../static/}}

\begin{document}

\maketitle

\section{Ejercicio 1}

\subsection{redistribucionDeLosFrutos}

\begin{proc}{redistribucionDeLosFrutos}{\In recursos: \TLista{\float}, \In cooperan: \TLista{\bool}}{\TLista{\float}}
	\requiere{\paraTodo[unalinea]{e}{\float}{e \in recursos \implica e > 0} \land \longitud{cooperan} = \longitud{recursos}}
	\asegura{|\res| = |recursos| \yLuego \\ \paraTodo[unalinea]{i}{\ent}{0 \leq i < |\res| \implicaLuego res[i] = \IfThenElse{cooperan[i] = \True}{\frac{fondoComun(recursos, cooperan)}{\longitud{recursos}}}{recursos[i] + \frac{fondoComun(recursos, cooperan)}{\longitud{recursos}}}}} 
\end{proc}

\aux{fondoComun}{recursos: \TLista{\float}, cooperan: \TLista{\bool}}{\float}{ \\\sum\limits_{i=0}^{\longitud{recursos}-1} \IfThenElse{cooperan[i] = \True}{recursos[i]}{0}
}

\subsection{trayectoriaDeLosFrutosIndividualesALargoPlazo}

\begin{proc}{trayectoriaDeLosFrutosIndividualesALargoPlazo}{
	\Inout trayectorias: \TLista{\TLista{\float}}, \In cooperan: \TLista{\bool}, \In apuestas: \TLista{\TLista{\float}},
	\In pagos: \TLista{\TLista{\float}}, \In eventos: \TLista{\TLista{\nat}}
}{}

	\requiere{\\
		|trayectorias| = |eventos| = |cooperan| = |apuestas| = |pagos| \yLuego \\
		\paraTodo[unalinea]{i}{\ent}{0 \le i < |cooperan| \implicaLuego |trayectorias[i]| = 1 \land |apuestas[i]| = |pagos[i]|} \land \\
		\paraTodo[unalinea]{i}{\ent}{0 \le i < |eventos| \implicaLuego |eventos[i]| = |eventos[0]|} \land \\
		todosElementosDeLasListasPositivos(pagos) \land \\ 
		todosElementosDeLasListasPositivos(trayectorias) \land \\
		todosElementosDeLasListasPositivos(apuestas) \land \\
		eventosValidos(eventos, pagos) \land \\seApuestaTodo(apuestas)\\
	}
	\requiere{\\
		\paraTodo[unalinea]{j}{\ent}{0 \le j < |apuestas| - 1 \implicaLuego |apuestas[j]| = |apuestas[j+1]|}\\
	}
	\requiere{\\
		\paraTodo[unalinea]{j}{\ent}{0 \le j < |pagos| - 1 \implicaLuego |pagos[j]| = |pagos[j+1]|}\\
	}

	\asegura{\\
		|trayectorias| = |eventos| \yLuego \\
		\paraTodo[unalinea]{i}{\ent}{0 \le i < |trayectorias| \implicaLuego |trayectorias[i]| = |eventos[0]|+1} \land \\
		elPrincipioSigueIgual(old(trayectorias), trayectorias) \land \\
		estaBienCalculado(trayectorias, eventos, cooperan, apuestas, pagos)\\
	}

\end{proc}

\pred{todosElementosDeLasListasPositivos}{pagos: \TLista{\TLista{\float}}}{
     \paraTodo[unalinea]{i}{\ent}{\paraTodo[unalinea]{t}{\ent}{0 \leq i<|pagos| \land 0\leq t <|pagos[i]|} \implicaLuego pagos[i][t] > 0}
}

\pred{eventosValidos}{eventos: \TLista{\TLista{\nat}}, pagos: \TLista{\TLista{\float}}}{
	\paraTodo[unalinea]{i}{\ent}{\paraTodo[unalinea]{t}{\ent}{0 \leq i<|eventos| \land 0\leq t <|eventos[i]| \implicaLuego 0 \leq eventos[i][t]<|pagos[i]|}}
}

\pred {seApuestaTodo}{apuestas:\TLista{\TLista{\float}}}{
    \paraTodo[unalinea]{i}{\ent}{0 \leq i<|apuestas| \implicaLuego \sum\limits_{t=0}^{|apuestas[i]|-1}{apuestas[i][t]}=1}}

\pred{elPrincipioSigueIgual}{oldtrayectorias: \TLista{\TLista{\float}}, trayectorias:\TLista{\TLista{\float}}}{
	\paraTodo[unalinea]{i}{\ent}{0 \leq i < |trayectorias|  \implicaLuego 
	oldtrayectorias[i][0] = trayectorias [i][0]}
}

\pred{estaBienCalculado}{trayectorias: \TLista{\TLista{\float}}, eventos: \TLista{\TLista{\nat}}, 
cooperan: \TLista{\bool}, apuestas: \TLista{\TLista{\float}}, \\ pagos: \TLista{\TLista{\float}}}{
	\paraTodo[unalinea]{i}{\ent}{\paraTodo[unalinea]{t}{\ent}{(0 \leq i < |cooperan| \land 0 \le t < |eventos[0]|) \\ \implicaLuego 
	trayectorias[i][t+1] = calcularPlataConFondo(i, t, cooperan, apuestas, pagos, eventos, trayectorias)}}
}

\aux{calcularPlataConFondo}{individuo: \nat, tiempo: \nat, cooperan: \TLista{\bool}, apuestas: \TLista{\TLista{\float}}, 
pagos: \TLista{\TLista{\float}}, eventos: \TLista{\TLista{\nat}}, trayectorias: \TLista{\TLista{\float}}}{\float}{ \\
	\IfThenElse{cooperan[individuo] = \True}{fondoDistribuido(tiempo, apuestas, pagos, eventos, trayectorias, cooperan) \\}
	{nuevaPlataSinFondo(individuo, tiempo, apuestas, pagos, eventos, trayectorias) \\
	+ fondoDistribuido(tiempo, apuestas, pagos, eventos, trayectorias, cooperan)}
}

\aux{nuevaPlataSinFondo}{individuo: \nat, tiempo: \nat, apuestas: \TLista{\TLista{\float}}, 
pagos: \TLista{\TLista{\float}}, eventos: \TLista{\TLista{\nat}}, \\ trayectorias: \TLista{\TLista{\float}}}
{\float}{\\
	apuesta[individuo][eventos[individuo][tiempo]] * pagos[individuo][eventos[individuo][tiempo]] \\ *trayectorias[individuo][tiempo]
}

\aux{fondoDistribuido}{tiempo: \ent, apuestas: \TLista{\TLista{\float}}, pagos: \TLista{\TLista{\float}}, 
eventos: \TLista{\TLista{\ent}}, trayectorias: {\TLista{\TLista{\float}}}, cooperan: \TLista{\bool}}{\float}{ \\
	\sum\limits_{i=0}^{|cooperan|-1}{\IfThenElse{cooperan[i] = \True \\}
	{\frac{nuevaPlataSinFondo(i, tiempo, apuestas, pagos, eventos, trayectorias)}{|cooperan|} \\}{0}}
}

\subsection{trayectoriaExtrañaEscalera}

\begin{proc}{trayectoriaExtrañaEscalera}{\In trayectoria: \TLista{\float}}{\bool}
	\requiere{|trayectoria| \geq 1}
	\requiere{\paraTodo[unalinea]{i}{\ent}{0 \le i < |trayectoria| \implicaLuego trayectoria[i] > 0}} %TODO: es mayor estricto??? (corrección TP)
	\asegura{\res = true \iff hayUnSoloMaximoLocal(trayectoria)} 
\end{proc}

\pred {hayUnSoloMaximoLocal}{trayectoria: \TLista{\float}}{
    \existe[unalinea]{i}{\ent}{0 \leq i<|trayectoria| \yLuego esMaximoLocal(i, trayectoria) \land \\ 
	\neg \existe[unalinea]{j}{\ent}{0 \leq j<|trayectoria| \yLuego j \neq i \land esMaximoLocal(j, trayectoria)}}}

\pred {esMaximoLocal}{i: \ent, trayectoria: \TLista{\float}}{
    |trayectoria| = 1 \oLuego \\
	(i = 0 \land trayectoria[0] > trayectoria[1]) \lor \\
	(i = |trayectoria| - 1 \land trayectoria[|trayectoria| - 1] > trayectoria[|trayectoria| - 2]) \lor \\
	(0 < i < |trayectoria| -1 \yLuego trayectoria[i - 1] < trayectoria[i] \land trayectoria[i] >trayectoria[i + 1])}

\subsection{individuoDecideSiCooperarONo}

\begin{proc}{individuoDecideSiCooperarONo}{\In individuo: \nat, \In recursos: \TLista{\float}, \Inout cooperan: \TLista{\bool},
 \In apuestas: \TLista{\TLista{\float}}, \In pagos: \TLista{\TLista{\float}}, \In eventos: \TLista{\TLista{\nat}}}{}
	\requiere{\\
		|pagos|=|apuestas|=|eventos|=|cooperan|=|recursos| \land 0 \le individuo < |cooperan| \yLuego \\
		\paraTodo[unalinea]{i}{\ent}{0 \le i < |apuestas| \implicaLuego |apuestas[i]| = |pagos[i]| \land |eventos[0]| = |eventos[i]| \land |eventos[i]| > 0} \yLuego \\
		eventosValidos(eventos, pagos) \land \\
		todosElementosDeLasListasPositivos(pagos) \land \\
		todosElementosDeLasListasPositivos(apuestas) \land \\
		seApuestaTodo(apuestas) \land \\
		\paraTodo[unalinea]{i}{\ent}{0 \le i < |recursos| \implicaLuego recursos[i] > 0} \\ 
		}
		\requiere{\\
			\paraTodo[unalinea]{j}{\ent}{0 \le j < |apuestas| - 1 \implicaLuego |apuestas[j]| = |apuestas[j+1]|}\\
		}
		\requiere{\\
			\paraTodo[unalinea]{j}{\ent}{0 \le j < |pagos| - 1 \implicaLuego |pagos[j]| = |pagos[j+1]|}\\
		}

		\asegura{ \\
		|old(cooperan)| = |cooperan| \yLuego \\
		soloSeActualizoElIndividuo(individuo, old(cooperan), cooperan) \land \\
		cooperan[individuo] = \\ \IfThenElse{ fondoDivididoEnTurno(|eventos[0]| - 1, eventos, cooperanSiCoopera(individuo, cooperan), recursos, pagos, apuestas) \\\ge 
		recursosDeNoCooperadorEnTurno(|eventos[0]| - 1, individuo, eventos, pagos, apuestas, \\ cooperanSiNoCoopera(individuo, cooperan), recursos)}
		{\True }
		{\False}\\
	}
	% TODO si no hay eventos (tipo |eventos[individuo]| = 0) qué hacemos? pq divide los recursos 
	% iniciales de los colaboradores y los suma, pero dudo que en el "turno -1" aplique el fondo

\end{proc}
\vspace{5mm}

\pred{soloSeActualizoElIndividuo}{individuo: \ent, cooperanV: \TLista{\bool}, cooperanN: \TLista{\bool}}{
	\paraTodo[unalinea]{i}{\ent}{0 \le i < |cooperanV| \land i \neq individuo \implicaLuego cooperanV[i] = cooperanN[i]}
}

\aux{fondoDivididoEnTurno}{turno: \ent, eventos: \TLista{\TLista{\ent}}, cooperan: \TLista{\bool}, 
recursos: \TLista{\float}, pagos: \TLista{\TLista{\float}}, apuestas: \TLista{\TLista{\float}}}{\float}{\\
recursosDeCooperadoresIniciales(cooperan, recursos) * (\prod\limits_{k=0}^{turno}{tasasDeCooperadores(k, cooperan, pagos, apuestas, eventos)})
}
\aux{cooperanSiCoopera}{individuo: \ent, cooperan: \TLista{\bool}}{\TLista{\bool}}{
	\\ \sub{cooperan, 0, individuo} \masmas \langle \True \rangle \masmas \sub{cooperan, individuo+1, |cooperan|}
}

\aux{cooperanSiNoCoopera}{individuo: \ent, cooperan: \TLista{\bool}}{\TLista{\bool}}{
	\\ \sub{cooperan, 0, individuo} \masmas \langle \False \rangle \masmas \sub{cooperan, individuo+1, |cooperan|}
}

\aux{tasasDeCooperadores}{turno: \ent, cooperan: \TLista{\bool}, pagos: \TLista{\TLista{\float}}, apuestas: \TLista{\TLista{\float}, eventos: \TLista{\TLista{\ent}}}}{}{
	\\ \sum\limits_{i=0}^{|cooperan|-1}{\IfThenElse{cooperan[i] = \True}{\frac{tasaEnTurno(turno, i, pagos, apuestas, eventos)}{|cooperan|}}{0}}
}

\aux{recursosDeCooperadoresIniciales}{cooperan: \TLista{\bool}, recursos: \TLista{\float}}{\float}{
	\\ \sum\limits_{i=0}^{|cooperan|-1}{\IfThenElse{cooperan[i] = \True}{\frac{recursos[i]}{|cooperan|}}{0}}
}

\aux{tasaEnTurno}{turno: \ent, individuo: \ent, pagos: \TLista{\TLista{\float}}, 
apuestas: \TLista{\TLista{\float}}, eventos: \TLista{\TLista{\ent}}}{\float}{ \\
	pagos[individuo][eventos[individuo][turno]]*apuestas[individuo][eventos[individuo][turno]]
}

\aux{recursosDeNoCooperadorEnTurno}{turno: \ent, individuo: \ent, eventos: \TLista{\TLista{\ent}}, 
pagos: \TLista{\TLista{\float}}, apuestas: \TLista{\TLista{\float}}, cooperan: \TLista{\bool}, recursos: \TLista{\TLista{\float}}}{}{\\
	recursos[individuo] * \prod\limits_{t=0}^{turno}{tasaEnTurno(t, individuo, pagos, apuestas, eventos)} + \\
	\sum\limits_{i=0}^{turno}{(fondoDivididoEnTurno(i, eventos, cooperan, recursos, pagos, apuestas)} *\\
 	\prod\limits_{j=i+1}^{turno}{tasaEnTurno(j, individuo, pagos, apuestas, eventos)} \text{)} 
}
\paragraph{Demostración}recursosDeNoCooperadorEnTurno\\
$w_k = recursos[k]$\\
$t_k = tasaEnTurno(k, individuo, pagos, apuestas, eventos)$\\
$f_k = fondoDivididoEnTurno(k, eventos, cooperan, recursos, pagos, apuestas)$\\
\\
Tenemos que $w_n = w_{n-1} * t_n + f_n$ y queremos probar que vale nuestra fórmula: \[w_n = w_0 * \prod\limits_{i=1}^{n}{t_i} + \sum\limits_{i=1}^{n}{(f_i * \prod\limits_{j=i+1}^{n}{t_j})}\]
\subparagraph{Caso Base}Probamos que vale para $n=0$
\[w_n = w_0 * \prod\limits_{i=1}^{n}{t_i} + \sum\limits_{i=1}^{n}{(f_i * \prod\limits_{j=i+1}^{n}{t_j})}\]
\[w_0 = w_0 * \prod\limits_{i=1}^{0}{t_i} + \sum\limits_{i=1}^{0}{(f_i * \prod\limits_{j=i+1}^{0}{t_j})}\]
\[w_0 = w_0 * 1 + 0\]
\[w_0 = w_0 \]
\subparagraph{Paso Inductivo}Asumimos que nuestra fórmula vale para $n-1$ y demostramos que entonces valdrá para $n$
\[w_n = w_{n-1} * t_n + f_n\]
\[w_n = (w_0 * \prod\limits_{i=1}^{n-1}{t_i} + \sum\limits_{i=1}^{n-1}{(f_i * \prod\limits_{j=i+1}^{n-1}{t_j})}) * t_n + f_n\]
\[w_n = w_0 * t_n * \prod\limits_{i=1}^{n-1}{t_i} + t_n * \sum\limits_{i=1}^{n-1}{(f_i * \prod\limits_{j=i+1}^{n-1}{t_j})} + f_n\]
\[w_n = w_0 * \prod\limits_{i=1}^{n}{t_i} + t_n * \sum\limits_{i=1}^{n-1}{(t_n * f_i * \prod\limits_{j=i+1}^{n-1}{t_j})} + f_n\]
\[w_n = w_0 * \prod\limits_{i=1}^{n}{t_i} + \sum\limits_{i=1}^{n}{(f_i * \prod\limits_{j=i+1}^{n}{t_j})}\]
$Q.E.D$

\subsection{individuoActualizaApuesta}

\begin{proc}{individuoActualizaApuesta}{\In individuo: \nat, \In recursos: \TLista{\float} , \In cooperan: \TLista{\bool} , 
	\Inout apuestas: \TLista{\TLista{\float}}, \In pagos: \TLista{\TLista{\float}} , \In eventos: \TLista{\TLista{\nat}} }
	
	\requiere{\\
		|pagos|=|apuestas|=|eventos|=|cooperan|=|recursos| \land 0 \le individuo < |cooperan| \yLuego \\
		\paraTodo[unalinea]{i}{\ent}{0 \le i < |apuestas| \implicaLuego |apuestas[i]| = |pagos[i]| \land |eventos[0]| = |eventos[i]| \land |eventos[i]| > 0} \yLuego \\
		eventosValidos(eventos, pagos) \land \\
		todosElementosDeLasListasPositivos(pagos) \land \\
		todosElementosDeLasListasPositivos(apuestas) \land \\
		seApuestaTodo(apuestas) \land \\
		\paraTodo[unalinea]{i}{\ent}{0 \le i < |recursos| \implicaLuego recursos[i] > 0} \\ 
		}
	\requiere{\\
		\paraTodo[unalinea]{j}{\ent}{0 \le j < |apuestas| - 1 \implicaLuego |apuestas[j]| = |apuestas[j+1]|}\\
		}
	\requiere{\\
		\paraTodo[unalinea]{j}{\ent}{0 \le j < |pagos| - 1 \implicaLuego |pagos[j]| = |pagos[j+1]|}\\
	}
	
	\asegura{\\
	esApuestaSimilar(individuo, apuestas, old(apuestas)) \yLuego \\
	esApuestaValida(individuo, apuestas, pagos) \yLuego \\
	\paraTodo[unalinea]{s}{	\TLista{\TLista{\float}}}{esApuestaSimilar(individuo , apuestas , s) \land esApuestaValida(individuo, s, pagos) \implicaLuego \\
	dineroFinal(individuo, recursos , cooperan , apuestas , pagos , eventos)  \\
	\geq dineroFinal(individuo, recursos , cooperan , s , pagos , eventos)}
	}
	


\end{proc}

\vspace{13mm}

\aux{dineroFinal}{i: \ent , recursos: \TLista{\float} , cooperan: \TLista{\bool}, apuestas: \TLista{\TLista{\float}} , 
pagos: \TLista{\TLista{\float}} , eventos: \TLista{\TLista{\nat}}}{\float} {\\
	\IfThenElse{(cooperan[i] = true)} {\\fondoDivEnTurno(|eventos[0]| - 1, eventos, cooperan, 
	recursos, pagos, apuestas)}{\\recursosDeNoCooperadorEnTurno(|eventos[0]| - 1, i, eventos, 
	pagos, apuestas, cooperan, recursos)}
}


\pred{esApuestaSimilar}{i: \nat, apuestas: \TLista{\TLista{\float}}, s: \TLista{\TLista{\float}}}{
	(|apuestas| = |s|) \yLuego \paraTodo[unalinea]{j}{\ent}{(0 \leq j < |apuestas|) \land  (j \neq i) \implicaLuego ( apuestas[j] = s[j] )}
}

\pred{esApuestaValida}{i: \ent , apuestas: \TLista{\TLista{\float}}, pagos: \TLista{\TLista{\float}}}{
	|apuestas[i]| = |pagos[i]| \land ( \sum\limits_{k=0}^{|apuestas|-1}{apuestas[i][k]} = 1 ) \land 
	\paraTodo[unalinea]{j}{\ent}{0 \leq j < |apuestas[i]| \implicaLuego apuestas[i][j] > 0}   
	% que vea que se apuesta todo, y que el largo sea igual al de pagos, y que todas las apuestas sean positivas (mayores exclusivas a 0?)
}

\section{Ejercicio 2}
\begin{flushleft}
Queremos probar  la validez de la siguiente tripla de Hoare:
\textbf{\{ P \}S\{ Q \}} la cual es válida si solo si \textbf{{P $\implicaLuego$ wp(S,Q)}.}\\                                  
\vspace{3mm}
\textbf{P (precondición)} $\equiv \{ apuesta_c + apuesta_s =1 \wedge pago_c>0 \wedge pago_s>0 \wedge apuesta_c>0 \wedge apuesta_s>0 \wedge recurso>0\}$\\
\vspace{3mm}
\textbf{Q (postcondición)} $\equiv$ \{res=recurso($apuesta_cpago_c)^{\#(eventos,T)}$($apuesta_spago_s)^{\#(eventos,F)}$\}\\
\vspace{3mm}
Nótese que \#$(eventos, T)$ es la notación abreviada del auxiliar {\normalfont\ttfamily\#apariciones$(eventos, T)$} 
que se definió en la teórica. \\
\vspace{3mm}
\textbf{S (programa)} $\equiv$
\begin{lstlisting}
	res := recurso;
	i := 0;
	while (i < eventos.size()) do
	    if eventos[i] then
		   res:= (res * apuesta.c)pago.c
		else
		   res:= (res * apuesta.s)pago.s
		endif
		i := i + 1
	endwhile
\end{lstlisting} 

\vspace{3mm}
\textbf{Dividimos el programa en tres partes (S1,S2,S3)}:\\
\begin{lstlisting}
	S1
	res := recurso;
	S2
	i := 0;
	S3
	while (i < eventos.size()) do
	     if eventos[i] then
		   res:= (res * apuesta.c)pago.c
		 else
		   res:= (res * apuesta.s)pago.s
		 endif
		i := i + 1
	endwhile
		\end{lstlisting} 

\vspace{3mm}
Luego, queriamos probar que P $\implicaLuego$ wp(S,Q). Esto es lo mismo que probar que P $\implicaLuego$ wp(S1;S2;S3,Q).\\
\vspace{3mm}
Primero debemos hallar \textbf{wp(S1;S2;S3,Q) $\equiv$ wp(S1;S2,wp(S3,Q))}.\\
\vspace{3mm}
\textbf{wp(S3,Q)} $\equiv$ PC (precondición del ciclo) $\equiv$ \{res = recurso $\wedge$ i=0\}\\
\vspace{3mm}
La PC la planteamos a partir de la definición del ciclo. 
Nuestras demostraciones no pueden asegurar que esta PC es la más débil, 
pero sí puede asegurar que implica a la \textbf{wp}, si se cumple la correctitud del ciclo.\\
\vspace{10mm}
\textbf{Demostración de la correcitud del ciclo:}\\
\vspace{3mm}
Para ello necesitamos: 
\begin{enumerate} \setlength\itemsep{0cm}
	\item \textbf{I (invariante)} $\equiv 0\leq i \leq |eventos| \wedge res=recurso(apuesta_cpago_c)^{\#(subseq(eventos,0,i),T)}(apuesta_spago_s)^{\#(subseq(eventos,0,i),F)}$
	\item \textbf{PC} $\equiv \{res= recurso \wedge i=0\}$
	\item \textbf{QC (postcondición del ciclo)} $\equiv \{res=recurso(apuesta_cpago_c)^{\#(eventos,T)}(apuesta_spago_s)^{\#(eventos,F)} \wedge i=|eventos|$\}
	\item \textbf{BC (guarda del ciclo)} $\equiv i < |eventos|$
	\item \textbf{fv (función variante)} $= |eventos|-i$
\end{enumerate}

Luego, hay que demostrar:
\begin{enumerate} \setlength\itemsep{0cm}
	\item \textbf{$PC \implica I$}
	\item \textbf{$\{ I \wedge B \}S\{ I \}$}
	\item \textbf{$I \wedge \neg B \implica QC$}
	\item \textbf{$\{ I \wedge B \wedge fv= v_0 \}S\{ fv<v_0 \}$}
	\item \textbf{$I \wedge fv\leq0 \implica \neg B$} 

\end{enumerate}

\begin{enumerate} \setlength\itemsep{4mm}
	\item \textbf{$PC \implica I$} \\
	\vspace{2mm}
	$res= recurso$ $\wedge$ i = 0 $\implica 0\leq i\leq|eventos| \wedge$ res = recurso$(apuesta_cpago_c)^{\#(subseq(eventos,0,i),T)}(apuesta_spago_s)^{\#(subseq(eventos,0,i),F)} \equiv$\\
	\vspace{3mm}
	\textbf{(Asumo que vale PC (luego vale cada termino que lo compone))}\\
	\vspace{3mm}
	\textbf{(Reemplazo en I a i por i=0 (ya que asumimos que vale))}\\
	\vspace{3mm}
	$res= recurso$ $\wedge$ i = 0 $\implica 0\leq 0\leq|eventos| \wedge$ $res=recurso$($apuesta_cpago_c)^{\#(subseq(eventos,0,0),T)}$($apuesta_spago_s)^{\#(subseq(eventos,0,0),F)} \equiv$\\
	\vspace{2mm}
	$res= recurso$ $\wedge$ i = 0  $\implica  True  \wedge$ $res=recurso$($apuesta_cpago_c)^{\#(\lvacia,T)}$($apuesta_spago_s)^{\#(\lvacia,F)} \equiv$ \\
	\vspace{2mm}
	$res= recurso$ $\wedge$ i = 0  $\implica$ $res=recurso$($apuesta_cpago_c)^{0}$($apuesta_spago_s)^{0} \equiv $\\
	\vspace{2mm}
	$res= recurso$ $\wedge$ i = 0  $\implica$ $res=recurso$ $\equiv $\\
	\vspace{2mm}
	$True$ \\
	\vspace{3mm}
	\textbf{Luego, queda demostrado que PC $\implica$ I}.\\

	\item \textbf{$\{ I \wedge B \}S\{ I \}$} \\
	\vspace{2mm}
	\textbf{Hay que probar que $\{ I \wedge B \} \implica wp(S,I)$}.\\
	\vspace{2mm}
	\textbf{Primero resuelvo wp(S,I)} \\
	\vspace{2mm}
	(S = S3 y dentro de S3 tenemos a Sif (parte del programa que contiene al IfThenElse) y a S4 = i:=i+1).\\
	\vspace{2mm}
	\textbf{Luego, wp(S,I) $\equiv$ wp(Sif;S4,I) $\equiv$ wp(Sif, wp(S4,I))} \\
	\vspace{6mm}
	\textbf{Resuelvo wp(S4,I)}\\
	\vspace{2mm}
	$wp(S4,I)\equiv wp(i:=i+1,I)$ $\equiv$\\
	\vspace{2mm}
	$def(i+1) \land I_{i+1}^{i}$ $\equiv$\\
	\vspace{2mm}
	$0 \leq i+1 \leq |eventos| \wedge res= recursos(apuesta_cpago_c)^{\# (subseq(eventos,0,i+1),T)}(apuesta_spago_s)^{\# (subseq(eventos,0,i+1),F)}$\\
	\vspace{2mm}
	\textbf{Denoto a wp(i:=i+1,I) como $"Qif"$} \\
	\vspace{6mm}
	\textbf{Resuelvo wp(Sif,Qif)}\\
	\vspace{2mm}
	$wp(Sif,Qif)$ $\equiv$\\
	\vspace{2mm}
	$def (eventos[i]) \land ((eventos[i]=true \wedge wp (res:=(res*apuesta_c)pago_c,Qif))\vee (eventos[i]=false \wedge wp (res:=(res*apuesta_s)pago_s,Qif)))$\\
	\vspace{6mm}
	\textbf{(Resuelvo $(eventos[i]=true \wedge wp (res:=(res*apuesta_c)pago_c,Qif))$) }\\
	\vspace{2mm}
	\textbf{(Reservo el $eventos[i]=true$ y resuelvo $wp (res:=(res*apuesta_c)pago_c,Qif)$)}\\
	\vspace{2mm}
	$wp (res:=(res*apuesta_c)pago_c,Qif)$ $\equiv$\\
	\vspace{2mm}
	$def(res:=(res*apuesta_c)pago_c) \land Qif_{(res*apuesta_c)pago_c}^{res}$ $\equiv$\\
	\vspace{2mm}
	$0 \leq i+1 \leq |eventos| \wedge (res*apuesta_c)pago_c= recursos(apuesta_cpago_c)^{\# (subseq(eventos,0,i+1),T)}(apuesta_spago_s)^{\# (subseq(eventos,0,i+1),F)}$ $\equiv$\\
	\vspace{2mm}
	$0 \leq i+1 \leq |eventos| \wedge (res*apuesta_c)pago_c= recursos(apuesta_cpago_c)^{\# (subseq(eventos,0,i),T)}(apuesta_cpago_c)^{\# (subseq(eventos,i,i+1),T)}$ \\
	\vspace{1mm}
	$(apuesta_spago_s)^{\# (subseq(eventos,0,i),F)}(apuesta_spago_s)^{\# (subseq(eventos,i,i+1),T)}$ $\equiv$\\
	\vspace{3mm}
	\textbf{(Sabiendo que $eventos[i]=true$, reordeno la equivalencia)}\\
	\vspace{2mm}
	$0 \leq i+1 \leq |eventos| \wedge (res*apuesta_c)pago_c= recursos(apuesta_cpago_c)^{\# (subseq(eventos,0,i),T)}(apuesta_cpago_c)^{1}(apuesta_spago_s)^{\# (subseq(eventos,0,i),F)}(apuesta_spago_s)^{0}$ $\equiv$\\
	\vspace{2mm}
	$0 \leq i+1 \leq |eventos| \wedge (res*apuesta_c)pago_c= recursos(apuesta_cpago_c)^{\# (subseq(eventos,0,i),T)}(apuesta_cpago_c)(apuesta_spago_s)^{\# (subseq(eventos,0,i),F)}$ $\equiv$\\
	\vspace{2mm}
	$0 \leq i+1 \leq |eventos| \wedge res= recursos(apuesta_cpago_c)^{\# (subseq(eventos,0,i),T)}(apuesta_spago_s)^{\# (subseq(eventos,0,i),F)}$ \\
	\vspace{8mm}
	\textbf{(Resuelvo $(eventos[i]=false \wedge wp (res:=(res*apuesta_s)pago_s,Qif))$) }\\
	\vspace{2mm}
	\textbf{(Reservo el $eventos[i]=false$ y resuelvo $wp (res:=(res*apuesta_s)pago_s,Qif)$)}\\
	\vspace{2mm}
	$wp (res:=(res*apuesta_s)pago_s,Qif)$ $\equiv$\\
	\vspace{2mm}
	$def(res:=(res*apuesta_s)pago_s) \land Qif_{(res*apuesta_s)pago_s}^{res}$ $\equiv$\\
	\vspace{2mm}
	$0 \leq i+1 \leq |eventos| \wedge (res*apuesta_s)pago_s= recursos(apuesta_cpago_c)^{\# (subseq(eventos,0,i+1),T)}(apuesta_spago_s)^{\# (subseq(eventos,0,i+1),F)}$ $\equiv$\\
	\vspace{2mm}
	$0 \leq i+1 \leq |eventos| \wedge (res*apuesta_s)pago_s= recursos(apuesta_cpago_c)^{\# (subseq(eventos,0,i),T)}(apuesta_cpago_c)^{\# (subseq(eventos,i,i+1),T)}$ \\
	\vspace{1mm}
	$(apuesta_spago_s)^{\# (subseq(eventos,0,i),F)}(apuesta_spago_s)^{\# (subseq(eventos,i,i+1),T)}$ $\equiv$\\
	\vspace{3mm}
	\textbf{(Sabiendo que $eventos[i]=false$, reordeno la equivalencia)}\\
	\vspace{2mm}
	$0 \leq i+1 \leq |eventos| \wedge (res*apuesta_s)pago_s= recursos(apuesta_cpago_c)^{\# (subseq(eventos,0,i),T)}(apuesta_cpago_c)^{0}(apuesta_spago_s)^{\# (subseq(eventos,0,i),F)}(apuesta_spago_s)^{1}$ $\equiv$\\
	\vspace{2mm}
	$0 \leq i+1 \leq |eventos| \wedge res= recursos(apuesta_cpago_c)^{\# (subseq(eventos,0,i),T)}(apuesta_spago_s)^{\# (subseq(eventos,0,i),F)}$ \\
	\vspace{6mm}
	\textbf{(En ambas resoluciones llegamos a la misma wp la cual denotamos como $"resIf"$ )} \\
	\vspace{6mm}
	\textbf{(Juntando todo)} \\
	\vspace{2mm}
	$def (eventos[i]) \land ((eventos[i]=true \wedge wp (res:=(res*apuesta_c)pago_c,Qif))\vee (eventos[i]=false \wedge wp (res:=(res*apuesta_s)pago_s,Qif)))$ $\equiv$\\
	\vspace{2mm}
	$0 \leq i < |eventos| \land ((eventos[i]=true \wedge resIf)\vee (eventos[i]=false \wedge resIf))$ $\equiv$\\
	\vspace{2mm}
	$0 \leq i < |eventos| \wedge resIf \land (eventos[i]=true \vee eventos[i]=false)$ $\equiv$\\
	\vspace{2mm}
	$0 \leq i < |eventos| \wedge res= recursos(apuesta_cpago_c)^{\# (subseq(eventos,0,i),T)}(apuesta_spago_s)^{\# (subseq(eventos,0,i),F)}$ \\
	\vspace{4mm}
	\textbf{Luego, wp(Sif,Qif) $\equiv$ $0 \leq i < |eventos| \wedge res= recursos(apuesta_cpago_c)^{\# (subseq(eventos,0,i),T)}(apuesta_spago_s)^{\# (subseq(eventos,0,i),F)}$}\\
	\vspace{6mm}
	\textbf{Ahora ya sabiendo cuánto vale wp(S,I) podemos probar la validez de $\{ I \wedge B \}S\{ I\}$, es decir, que $\{ I \wedge B \} \implica wp(S,I)$} \\
	\vspace{2mm}
	$I \wedge B \implica wp(S,I)$ $\equiv$\\
	\vspace{2mm}
	\textbf{(Para simplificar escribo a I como $0\leq i \leq |eventos| \wedge res_i$)}\\
	\vspace{2mm}
	$0\leq i \leq |eventos| \wedge res_i \wedge $ $i < |eventos| \implica wp(S,I)$ $\equiv$\\
	\vspace{2mm}
	$0\leq i <|eventos| \wedge res_i \implica wp(S,I)$ \\
	\vspace{2mm}
	\textbf{Esto es verdad ya que wp(S,I) $\equiv$ $0\leq i <|eventos| \wedge resIf $ ($resIf=res_i$)}\\
	\vspace{4mm}
	\textbf {Luego, queda demostrado que $\{ I \wedge B \}S\{ I \}$ vale.}

	\item \textbf{$I \wedge \neg B \implica QC$}\\
	\vspace{2mm}
	$0\leq i \leq |eventos| \wedge res=recurso(apuesta_cpago_c)^{\#(subseq(eventos,0,i),T)}(apuesta_spago_s)^{\#(subseq(eventos,0,i),F)} \wedge i \geq |eventos| \implica QC \equiv $ \\
	\vspace{2mm}
	\textbf{(Como $i \leq |eventos| \wedge i\geq |eventos|$, luego $i=|eventos|$)}\\
	\vspace{2mm}
	$i = |eventos| \wedge res=recurso(apuesta_cpago_c)^{\#(subseq(evento,0,i),T)}(apuesta_spago_s)^{\#(subseq(eventos,0,i),F)}  \implica QC \equiv $ \\
	\vspace{3mm}
	\textbf{(Asumo que vale ($I \wedge \neg B$) (luego vale cada termino que lo compone))}\\
	\vspace{3mm}
	\textbf{(Reemplazo en ($I \wedge \neg B $) a i por $i= |eventos|$ (ya que asumimos que vale))}\\
	\vspace{3mm}
	$i = |eventos| \wedge res=recurso(apuesta_cpago_c)^{\#(subseq(eventos,0,|eventos|),T)}(apuesta_spago_s)^{\#(subseq(eventos,0,|eventos|),F)} \implica QC \equiv $ \\
	\vspace{2mm}
	$ i=|eventos| \wedge res=recurso(apuesta_cpago_c)^{\#(eventos,T)}(apuesta_spago_s)^{\#(eventos,F)} \implica QC \equiv $ \\
	\vspace{2mm}
	$QC \implica QC$ \\
	\vspace{3mm}
	\textbf{Luego, queda demostrado que $I \wedge \neg B \implica QC$}. \\


	\item \textbf{$\{ I \wedge B \wedge fv= v_0 \}S\{ fv<v_0 \}$}\\
	\vspace{2mm} 
	\textbf{Hay que probar que $\{ I \wedge B \wedge fv= v_0 \} \implica wp(S,fv<v_0)$}.\\
	\vspace{2mm}
	\textbf{Resuelvo primero wp(S,$fv<v_0$)}\\
	\vspace{2mm}
	(S = S3 y dentro de S3 tenemos a Sif (parte del programa que contiene al IfThenElse) y a S4 = i:=i+1).\\
	\vspace{2mm}
	\textbf{Luego, wp(S,$fv<v_0$) $\equiv$ wp(Sif;S4,$fv<v_0$) $\equiv$ wp(Sif, wp(S4,$fv<v_0$))} \\
	\vspace{6mm}
	\textbf{Resuelvo $wp (S4,fv<v_0)$}\\
	$wp(S4,fv<v_0) \equiv wp(i:= i+1,|eventos|-i <vo) \equiv def (i+1) \land (|eventos|-i<vo)_{i+1}^{i} \equiv$ \\
	\vspace{2mm}
	$True \land |eventos|-i-1<v_0$  $\equiv |eventos|-i-1<v_0$ \\
	\vspace{6mm}
	\textbf{Resuelvo wp(Sif,$|eventos|-i-1<v_0$)}\\
	\vspace{2mm}
	wp(Sif,$|eventos|-i-1<v_0$) $\equiv$ \\
	\vspace{2mm}
	$def (eventos[i]) \land ((eventos[i]=true \wedge wp (res:=(res.apuesta_c)pago_c,|eventos|-i-1<v_0))\vee (eventos[i]=false \wedge wp (res:=(res.apuesta_s)pago_s,|eventos|-i-1<v_0)))$\\
	\vspace{2mm}
	\textbf{(Resuelvo $(eventos[i]=true \wedge wp (res:=(res.apuesta_c)pago_c,|eventos|-i-1<v_0))$)} \\
	\vspace{2mm}
	$eventos[i]=true \wedge wp (res:=(res.apuesta_c)pago_c,|eventos|-i-1<v_0) $ $\equiv$ \\
	\vspace{2mm}
	$eventos[i]=true \wedge  |eventos|-i-1<v_0$ \\
	\vspace{2mm}
	\textbf{(Resuelvo $(eventos[i]=false \wedge wp (res:=(res.apuesta_s)pago_s,|eventos|-i-1<v_0))$ )} \\
	\vspace{2mm}
	$eventos[i]=false \wedge wp (res:=(res.apuesta_s)pago_s,|eventos|-i-1<v_0) $ $\equiv$ \\
	\vspace{2mm}
	$eventos[i]=false \wedge  |eventos|-i-1<v_0$ \\
	\vspace{2mm}
	\textbf{(Juntando todo)} \\
	\vspace{2mm}
	$def (eventos[i]) \land ((eventos[i]=true \wedge wp (res:=(res.apuesta_c)pago_c,|eventos|-i-1<v_0))\vee (eventos[i]=false \wedge wp (res:=(res.apuesta_s)pago_s,|eventos|-i-1<v_0)))$ $\equiv$ \\
	\vspace{2mm}
	$0 \leq i <|eventos| \land$ (($eventos[i]=true \wedge  |eventos|-i-1<v_0$) $\vee$  ($eventos[i]=false \wedge  |eventos|-i-1<v_0$)) $\equiv$\\
	\vspace{2mm}
	$0 \leq i <|eventos| \wedge$ $|eventos|-i-1<v_0 \wedge (eventos[i]=true \vee eventos[i]=false)$ $\equiv$ \\
	\vspace{2mm}
	$0 \leq i <|eventos| \wedge$ $|eventos|-i-1<v_0 \wedge True$ $\equiv$ \\
	\vspace{2mm}
	$0 \leq i <|eventos| \wedge$ $|eventos|-i-1<v_0$ \\
	\vspace{2mm}
	\textbf{Luego, wp(S,$fv<v_0$) $\equiv$ $0 \leq i <|eventos| \wedge$ $|eventos|-i-1<v_0$}\\
	\vspace{6mm}
	\textbf{Ahora ya sabiendo cuanto vale wp(S,$fv<v_0$) podemos probar la validez de $\{ I \wedge B \wedge fv= v_0 \}S\{ fv<v_0 \}$, es decir, que $\{ I \wedge B \wedge fv= v_0 \} \implica wp(S,fv<v_0)$} \\
	\vspace{2mm}
	$\{ I \wedge B \wedge fv= v_0 \} \implica wp(S,fv<v_0)$ $ \equiv$ \\
	\vspace{2mm}
	\textbf{(Para simplificar escribo a I como $0\leq i \leq |eventos| \wedge res_i$)}\\
	\vspace{2mm}
	$0\leq i \leq |eventos| \wedge res_i \wedge $ $i < |eventos|$ $\wedge |eventos|-i = v_0 \implica 0 \leq i <|eventos| \wedge$ $|eventos|-i-1<v_0$ $\equiv$ \\
	\vspace{2mm}
	$res_i \wedge 0\leq i<|evento| \wedge |eventos|-i = v_0 \implica 0 \leq i <|eventos| \wedge$ $|eventos|-i-1<v_0$ $\equiv$ \\
	\vspace{2mm}
	\textbf{(Como $0 \leq i <|eventos|$ está en el antecedente y en el consecuente, lo puedo sacar del consecuente)}\\
	\vspace{2mm}
	$res_i \wedge 0\leq i<|evento| \wedge |eventos|-i = v_0 \implica$ $|eventos|-i-1<v_0$ $\equiv$ \\
	\vspace{2mm}
	\textbf{(Asumo que vale ($I \wedge B \wedge fv= v_0$) y reemplazo a $v_0$ por fv)}\\
	\vspace{2mm}
	$|eventos|-i-1<|eventos|-i$ $\equiv$ \\
	\vspace{2mm}
	$-1<0 $ $\equiv$ \\
	\vspace{2mm}
	$True $ \\
	\vspace{4mm}
	\textbf{Luego, queda demostrado que $\{ I \wedge B \wedge fv= v_0 \}S\{ fv<v_0 \}$ vale.}

	\item \textbf{$I \wedge fv\leq0 \implica \neg B$} \\
	\vspace{2mm}
	$0\leq i \leq |eventos| \wedge res_i \wedge |eventos|-i \leq 0 \implica i \geq |eventos|$ $\equiv$ \\
	\vspace{2mm}
	$0\leq i \leq |eventos| \wedge res_i \wedge |eventos| \leq i \implica i \geq |eventos|$ $\equiv$ \\
	\vspace{2mm}
	$ res_i \wedge i= |eventos| \implica i \geq |eventos|$ $\equiv$ \\
	\vspace{2mm}
	\textbf{(Asumo que vale ($I \wedge fv\leq0$)) }\\
	\vspace{2mm}
	\textbf{(Reemplazo a i por $i=|eventos|$ ya que asumimos que vale)}\\
	\vspace{2mm}
	$res_i \wedge i=|eventos| \implica |eventos| \geq |eventos|$ $\equiv$ \\
	$True$ \\
	\vspace{3mm}
	\textbf{Luego, queda demostrado que $I \wedge fv\leq0 \implica \neg B$.}
\end{enumerate}
\textbf{Finalmente queda demostrada la correcitud del ciclo.}\\
\vspace{6mm}
\textbf{Volviendo al principio, queriamos demostrar}:\\
\vspace{2mm}
$P \implica wp(S,Q) \equiv P\implica wp(S1;S2;S3,Q)$ $ \equiv$\\
\vspace{2mm}
$P\implica wp(S1;S2,wp(S3,Q)) \equiv P\implica wp(S1;S2,PC) \equiv P\implica wp(S1,wp(S2,PC)) $\\
\vspace{4mm}
\textbf{Resuelvo wp(S1,wp(S2,PC))}\\
\vspace{4mm}
\textbf{Resuelvo primero wp(S2,PC)}\\
\vspace{2mm}
$wp(S2,PC) \equiv wp(i:=0,i=0 \wedge res=recurso)$ $\equiv$\\
\vspace{2mm}
$def (0) \wedge PC_{0}^{i}$ $\equiv$\\
\vspace{2mm}
$True \wedge 0=0 \wedge res=recurso$ $\equiv$\\
\vspace{2mm}
$res=recurso$\\
\vspace{4mm}
\textbf{Resuelvo wp(S1,res=recurso)}\\
\vspace{2mm}
$wp(res:=recurso,res=recurso)$ $\equiv$\\
\vspace{2mm}
$def(recurso) \wedge (res=recurso)_{recurso}^{res}$ $\equiv$\\
\vspace{2mm}
$recurso=recurso$ $\equiv$\\
\vspace{2mm}
$True$\\
\vspace{4mm}
\textbf{Luego wp(S,Q) $\equiv$ True}\\
\vspace{4mm}
\textbf{Ahora ya sabiendo cuanto vale wp(S,Q) podemos probar la validez de \{ P \}S\{ Q \}, es decir, que $P \implica wp(S,Q)$ vale.} \\
\vspace{2mm}
$P \implica wp(S,Q)$ $\equiv$\\
\vspace{2mm}
$P \implica True$\\
\vspace{4mm}
\textbf{Luego, queda demostrado que \{ P \}S\{ Q \} vale.}


\end{flushleft}

\end{document}
